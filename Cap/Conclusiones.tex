\chapter{Conclusiones}

El proyecto fue concebido como un proyecto estudiantil de extensión universitaria, lo cual puede apreciarse de acuerdo a sus objetivos y a su metodología de trabajo. A lo largo de todo el proyecto se buscó la participación activa de los colectivos, tanto en la generación de insumos para la sistematización de saberes, como para la evaluación crítica. Se logró una apropiación del proyecto por parte de la mayoría de los colectivos, lo que implica que el resultado final del mismo no es un producto exclusivo del equipo universitario, sino que es una construcción conjunta.\\

Asimismo, al trabajar sobre los aspectos que hacen a la gestión de las radios, se construyó un registro del estado de situación de cada una de ellas. Se hizo patente que la sistematización tenía valor desde el punto de vista histórico, dado que resume el panorama de las radios asociadas a AMARC en una coyuntura muy particular, de transición de una etapa de ilegalidad a otra, donde las radios adquieren derechos y obligaciones desde el punto de vista jurídico.\\

Además, aunque la intención original era que el producto del objetivo específico de sistematización de saberes fuera una especie de “manual de radiodifusión, para y por radialistas”, finalmente resultó ser demasiado ambicioso. La metodología de trabajo con cada colectivo, implicaba cubrir en una sola instancia muchos de los aspectos de la gestión de las radios, desde lo técnico a lo comunicacional, desde los recursos financieros a lo organizativo. De cada colectivo se logró obtener una visión global, validada como representativa en la instancia de plenario local. Pero, la amplitud de los temas impidió ahondar mucho en cada punto, con lo cual más que manual se terminó generando un conjunto de documentos con el estado de situación de las radios. Quizá también se puede diagnosticar que se partió de una hipótesis errónea en la formulación del proyecto, la cual era que los integrantes de las radios eran conscientes de los saberes que tenían. La experiencia de años de trabajo genera aprendizajes, pero estos pueden ser tan internalizados por los actores, que pueden llegar a pasar al terreno de lo cotidiano y no relevante.\\

Sin lugar a dudas, el estudio de audiencia es un hito en la historia de las radios comunitarias de Uruguay, dado que es el primer trabajo de alcance nacional en el tema.\\

Se puede discutir hasta qué punto se logró aportar en el proceso de fortalecimiento de las radios. El estudio de audiencia, con sus salvedades metodológicas, constituye un insumo fundamental para diseñar estrategias comunicacionales por parte de la red. Cabe destacar que, además de los resultados de cada pregunta, AMARC cuenta con la digitalización de todas las encuestas (una tabla, en formato .ods), lo que le permite realizar cruzamientos entre los puntos de las mismas, generando nuevas lecturas de la información relevada. En cuanto al objetivo de sistematización y colectivización de saberes, aunque los documentos generados no tienen la profundidad inicialmente pretendida, tienen el valor de servir como mojón en el proceso de fortalecimiento. Presentan un estado de situación de las radios a nivel nacional, en un amplio rango de temas, siendo un insumo importante para la toma de decisiones de la red. Ésa fue la evaluación que realizaron los participantes del taller nacional, que además valoraron mucho que el producto de la sistematización del relevamiento fuera colectivizado, para su edición crítica y participativa.