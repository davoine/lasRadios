\chapter{Introducción a la sistematización y colectivización de saberes}

Los siguientes documentos fueron realizados en base a los conocimientos que cada radio ha acumulado, a lo largo de su historia, en relación a su propia gestión. Incluye también parte de la historia de cada radio en sus aspectos fundacionales.\\

En la recolección de la información, así como en la discusión de los documentos, se trabajó en base a técnicas participativas, debido a que éstas permiten la discusión y reflexión colectiva. En base al conocimiento individual, los registros y memorias de cada integrante es que se constituye también la memoria colectiva, y en el trabajo en conjunto se potencia el conocimiento de todos.\\

El trabajo previsto en el encuentro con cada colectivo incluía la división de las tareas en equipos, cada uno con una temática diferente, y luego la realización de una instancia plenaria, donde cada grupo expone al resto de los participantes a efectos de complementar la información y realizar la síntesis colectiva de cada tema. En los casos donde los colectivos eran reducidos en número, no se realizó la división en equipos, trabajando en modalidad plenaria desde el principio.\\

Se utilizaron las mismas técnicas (una para cada área a relevar) en cada radio, para homogeneizar su sistematización. De esta forma fue posible una mejor organización de la información total relevada a nivel de toda la red.\\

Como referencia se tomó también el cuaderno bitácora (del equipo universitario) donde se registraba cada encuentro, para aclarar o complementar información.\\

Los documentos que siguen a continuación son una sistematización posible del vasto material recogido. Es la organización más adecuada, respetuosa y clara que el equipo universitario encontró para la presentación del material.\\

Todos los documentos fueron presentados a la red, trabajados en el taller nacional, discutidos y aprobados en la instancia plenaria. Posteriormente se recogieron los aportes y modificaciones sugeridas por los colectivos en dicha instancia.