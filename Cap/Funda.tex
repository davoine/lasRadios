\chapter{Fundaciones de las radios}
\indent Con cada colectivo se trabajó el tema ``Historia de la red'', buscando una descripción de la trayectoria de cada uno, mediante la elaboración colectiva de una narración. En este capítulo transcribiremos textualmente lo que cada colectivo generó, aunque (por motivos de espacio) sólo se extrae lo referente a la creación de la radio, a su concepción inicial, su fundación.\\

\section*{El Capiz (Valizas)}
``Yo diría que la movida de la radio empezó con un transmisor muy elemental hecho por Daniel (hermano del Cholo) que transmitía solo enchufándole un Walkman y salíamos con una radio a ver cuántas cuadras se escuchaba.

Un vecino que frecuentaba el pueblo y formaba parte de El Capiz y tenía contactos con AMARC, nos trajo un trasmisor estéreo y creamos La Marea FM, la cual funcionó unos dos años con mucha participación, y mucha desorganización.

Algunos integrantes de aquella primera radio seguimos con la idea de hacer radio y en el año 2006, el Cholo compró otro trasmisor y empezamos con El Capiz FM 107.5.

Empezó trasmitiendo en una piecita en la casa del Cholo, sólo los viernes de noche y domingos de mañana. Con un huevito y otro equipo musical.''

\section*{Horizonte (Paysandú)}
``Nace de una inspiración de Orlando, al ver pasar por el frente de su casa un chico drogándose. Al ser parte del barrio, sintió que debía hacer algo.
Entonces nace Horizonte FM, una radio comunitaria, la primera de Paysandú, con ganas de cambiarle la cabeza a la sociedad.

Como anécdota, cuando se fundó la radio había dos nombres para ponerle: Horizonte y Ombú (haciendo referencia al árbol más antiguo del barrio). Finalmente, por votación se escogió el nombre de Horizonte.

El 29 de marzo del año 2005, la radio sale al aire por primera vez, a las 19:30, con el seudónimo de Horizonte 989, emitiéndose desde un taller, con la ayuda de Gabriel.''

\section*{La Heladera (José Pedro Varela)}
``Un grupo de soñadores Varelenses de edades y ocupaciones diversas, comparten un proyecto: crear una radio como espacio de comunicación no comercial, con fines estéticos, culturales y de información objetiva a una comunidad bombardeada por los medios masivos y para ello se lanzan a la aventura de crear una radio comunitaria. El 23 de setiembre de 2005 con un equipo básico, en el living de una casa de familia, emitiendo tan solo 8 horas por día, se inician en el viaje que hoy ya es una realidad objetiva y no un sueño. Dos arduas tareas les esperaban, conseguir un espacio físico permanente y adecuado, y obtener la personería jurídica de la Asociación Civil que debería servir de sustento legal a la radio. La Asociación Civil se llamó El Candil y se obtuvo la personería jurídica en 2008, mientras que el local lo proporcionara la educadora insigne, Maestra Bimba Vega, quien por mucho tiempo sustenta gastos esenciales para el funcionamiento de la radio desde el 2006. (...)

Un hecho anecdótico interesante lo constituye la denominación de la radio: cuando nos instalamos en el local de Bimba, una vieja casona que había sido comité político, sus puertas no cerraban bien, y ante el temor de perder los equipos, éstos se guardaban en una vieja heladera en desuso, que se cerraba con cadena y candado. Con el paso del tiempo se producían diálogos tales como: “vamos a abrir la heladera” para referirse a que iba a sacar los equipos y comenzar la emisión… de ahí el nombre de la radio La Heladera.''

\section*{Parque (La Paloma)}
``La radio comenzó a salir desde un barrio de la Paloma (barrio Parque) como fruto del interés de un grupo de personas.

Su origen es anterior a la existencia de la ley. Luego fue inscripta en el censo y hoy está autorizada a emitir pero aun no está la resolución final de la URSEC.
El grupo inicial no mantuvo el trabajo de la radio y por tanto dejo de salir al aire durante un buen tiempo (tal vez un año).

A mediados del año 2009 surgió la idea de trasladar la radio a un gimnasio y de enganchar al liceo y otros centros para trabajar en la misma. A partir de allí se involucró un grupo nuevo de gente que es la que está hoy tratando de concretar el proyecto de radio comunitaria.''

\section*{Prado (Paso de la Arena)}
``El día 31/12/04 se inicia la radio con la primer salida 10 y 10 de la noche.

El 26 de febrero de ese año el agradecimiento a la gente, en el club Caffa, superando la venta de entradas de carnaval de ese establecimiento.

En agosto se festeja el día del niño llevando a los vecinos al parque Lecocq y compartiendo una fiesta para los niños, llenando la calle en ese día.

El primer año se realiza en la calle haciendo el primer reparto de juguetes, se sigue la relación con la gente, interactuando con los vecinos en los diferentes ámbitos. Tanto en ayuda con los merenderos, así con la ayuda en sillas de ruedas y bastones hasta mesas o canastas comestibles.''

\section*{Universo (Montes)}
``Dos jóvenes DJ de Montes un día a la venida de un baile, se les ocurrió de poner una radio para pasar música. Con la ayuda del abuelo de uno de ellos armaron un trasmisor artesanal, con una antena casera.

Se hacen las pruebas técnicas y los jóvenes quedan haciendo un programa musical que tuvo mucho éxito, por ser una novedad que estaban al aire y la gente llamaba para pedir temas.

Todo fue en una pieza de la estación de AFE por colaboración del padre de uno de los jóvenes. Luego se trata de formar más programas en la radio, invitando a personas vinculadas a temas sociales de la zona.''

\section*{Insomnio (Las Piedras)}
``La radio surge en octubre de 2006, ante la necesidad de profundizar el contacto con la comunidad, relación que se venía dando desde la fundación de la ONG en el año 1993.

2006 - inicio de actividades en el domicilio de un compañero, con pocos programas, muy poco conocimiento técnico y potencialidades de un medio de comunicación masivo.

2007 - un grupo de 9 compañer@s comienzan a capacitarse para el manejo global de la radio. Se inician contactos con AMARC afiliándose a la misma.''

\section*{La Cotorra (Cerro)}
``Había una vez un grupo de 39 compañeros/as que se fueron de un proyecto de radio y fundaron «La Cotorra». Después algunos/as se fueron y otros continúan hasta hoy. La idea continúa y sigue firme, tratando de mejorar día a día.

Todos los aportes de los compañeros/as continúan. Se ha logrado una continuidad aunque las personas cambien. El proyecto va más allá de las personas que los compongan: «La Cotorra cambió de frecuencia pero no de esencia». Teniendo siempre la comunicación con la comunidad. Hemos podido crecer técnicamente y en la gestión de la radio. Gracias al esfuerzo de todos/as, el apoyo del barrio que nos reconoce como un referente. Siempre dispuestos a apoyar emprendimientos barriales, tender redes y brindar ayuda.''

\section*{Espika (Santa Lucía)}

``En un pueblito no muy lejano (a 60 kilómetros de la capital) un grupo de amigos, preocupados por las guerras, decide cambiar el mundo. Surge entonces en 2003 la idea de un colectivo “Levantad Juventud”, apoyándose en los pricipios de participación popular, apuntando a la promoción de derechos humanos, haciendo foco en la equidad de género. Utilizando como ejes centrales un boletín y una radio “pirata” como fuente de difusión. 

En 2004 trasmitiendo al mundo y sus alrededores desde un garage. En esta faceta se ingresa a AMARC. En 2005 se plantea una nueva forma de trabajo: las comisiones. Posteriormente (2006) se pasa de un garage como estudio a un almacén, generando un lugar propio del colectivo.''

\section*{General Artigas (Toledo)}
``La historia de la radio se remonta a varios años antes de que ésta hiciera su primera salida al aire, ya que desde el año 1999 un pequeño grupo de inquietos de Toledo tenía la idea de poder contar en nuestra localidad una radio comunitaria. El por qué era que en la zona no había ningún medio de comunicación local, las radios y los canales de televisión son de Montevideo o son de otras ciudades, de la capital Canelones, de Pando o de Las Piedras y acá no había nada, ni un diario ni un semanario ni una revistita ni ningún medio de ningún tipo que informara de las cosas que pasaban en la ciudad. (...) Regularmente podíamos recibir en nuestra ciudad de Toledo las emisiones de dos radios comunitarias históricas del departamento como son la FM Tapié de San Ramón y la FM del Libertador de Barros Blancos. A partir de las escuchas fuimos pensando como poder concretar un medio de comunicación, que así como esas radios referidas difundían lo propio de su localidad nosotros también pudieramos cumplir con ese objetivo en Toledo.

Es así que, a finales del año 2001, comenzamos contactos con gente que estaba en el tema, como propiamente los compañeros de esas radios comunitarias nombradas (...)
% 
% Gracias al aporte de José (), luego de hacer una evaluación local por nuestra parte de cuales eran los canales disponibles, (en aquel tiempo habían muchos canales disponibles contrariamente a los que pasa hoy en día), comenzamos con la gestión de la construcción del transmisor el cual estuvo pronto a mediados del mes de agosto. Fueron en ese momento que comenzamos a colgar nuestra señal de aire con spots como por ejemplo éste. El hecho del inminente nacimiento de un medio local causo conmoción en nuestra comunidad de Toledo y eso se esparció como un reguero de pólvora afortunadamente.

En efecto en una fecha muy importante para nuestra patria, el 25 de agosto inauguramos la radio oficialmente a las 7 de la mañana y ya contábamos con algunos programas de producción local. Es de destacar que en esos primeros tiempos la radio no contaba ni con una estructura técnica ni con una infraestructura técnica ni con una infraestructura edilicia para desarrollar programas en vivo, debido a esto y a la voluntad de un montón de gente, se crearon en Toledo alrededor de 3 (entre comillas) estudios de grabaciones montados precariamente en casas de familia donde se generaban los programas, estos se grababan en disco compacto y se emitían a través de la radio.''

\section*{Timbó (San José)}

``Timbó nace de la necesidad de expresar y compartir por medio de la palabra, historias y vivencias no conocidas en la comunidad. No es solamente el deseo de ``hacer radio'' sino el deseo de hacer algo diferente desde una radio.

¿Por qué una radio?

Porque una radio produce el rescate de la palabra menoscabada, en épocas como esta. Abre las puertas de la imaginación, estimulando la creación por parte de los escuchas de nuevas imágenes y sentidos. Promueve entonces la riqueza del intercambio y la inclusión de lo singular. Habrá tantos programas como escuchas. Es crear símbolos, sonidos, es construir y producir. Estimula el deseo de hacer y pensar, ineludibles para vivir siendo protagonistas. La carta de las radios comunitarias y ciudadanas de AMARC dice: ``la palabra nos aproxima, nos revela, nos desarrolla, nos hace mejores hombres y mujeres. La palabra, libremente expresada, nos humaniza.''

\section*{Vilardevoz (Reducto)}
``Vos sabés qué es Vilardevoz.

Es nada más ni nada menos que la voz de los sin voz. Es el gritar estar vivo y tener derecho a pesar de que la circunstancia, te exilian, a veces; de la vida, de tu familia… de vos mismo y sobre todo el camino de regreso de un barco que va de lo imposible, a lo posible. De la ``utopía'' entre comillas absurda de un mundo enajenado, discriminado y aparte de muchos al derecho humano de la ``inclusión'' social con todos los derechos que todo ser humano debe usufructuar constitucionalmente: SALUD, VIVIENDA, TRABAJO. Que son la antítesis de ENFERMEDAD, INDIGENCIA y ABANDONO.

Esos derechos, son la voz de esta radio donde al encontrar soluciones se acallan las voces y se bajan los dedos discriminadores.``

\section*{Utopía (Colonia Nicolich)}

''A mediados del año 2006 en nuestro barrio Empalme Nicolich crecía una radio comunitaria. Entre mates y empanadas salió el nombre de Utopía (utopía era un sueño inalcanzable).

De a poquito entre amigos y vecinos fuimos conociéndonos, los mismos nos fueron prestando material discográfico para difundir en nuestra modesta radio.
Se aproximaba febrero y nuestro barrio no tenía su propio desfile de carnaval, con la ayuda brindada del municipio, de las agrupaciones barriales ``101 tambores'' y ``Copacabana'' y los compañeros que se disfrazaron.``
