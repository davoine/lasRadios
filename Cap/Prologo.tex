\phantomsection
\addcontentsline{toc}{chapter}{\protect\numberline{}Pr\'ologo}
\chapter*{Prólogo}
\begin{flushright}
\begin{small}
  \textit{``La radio que construimos es una donde quepan todos los pueblos y sus lenguas, que todos los pasos la caminen, que todos la rían, que la amanezcan todos.''}\\
(Parafraseando al sub-comandante Marcos)\end{small}
\end{flushright}

\vspace*{1cm}
``Las Radios no son Ruido. Experiencias Comunitarias Colectivizadas en Uruguay'', es la afirmación del título de este libro, que sintetiza un excelente trabajo de investigación, encuesta y entrevista que María Noel Sosa, Ángela Garofali, Pablo Hansen y Federico Davoine, rescatando las mejores tradiciones de la Universidad uruguaya y latinoamericana de puertas abiertas, han realizado como trabajo de extensión, como aporte para el estudio y experiencia de vida que no olvidarán ni los integrantes de las radios comunitarias visitadas ni -seguramente- los propios autores.\\

La Universidad comprometida con su gente, con su pueblo, allí donde esté viviendo, y con las formas organizativas que cada colectivo se dé a sí mismo, aprendiendo de la realidad y enseñándonos a vernos como realmente somos, descubriendo juntos estrategias para mejorar y lograr lo que nos proponemos.\\

Gracias a esta generación de jóvenes generosos y comprometidos, que entienden la comunicación como el rescate de las voces, las lenguas, los pasos, las risas de cada nuevo amanecer. Gracias por este esfuerzo, este excelente libro.\\

Las radios no son ruido, esta afirmación tiene la contundencia de respuesta. En el análisis de los datos obtenidos, en el descubrimiento de las capacidades de cada proyecto, de cada colectivo podrán descubrir allí las razones que explican esta afirmación, las radios no son ruido, al menos éstas no lo son ni quieren serlo.\\

Para el colectivo de un movimiento en red como es la Asociación Mundial de Radios Comunitarias (AMARC) Uruguay, también ha sido una experiencia rica y enriquecedora, que nos permitió mirarnos con una perspectiva nueva, que muestra con mayor objetividad fortalezas y debilidades, capacidades y errores. En la medida que estos jóvenes universitarios fueron involucrando los colectivos de las radios comunitarias entrevistadas y sus respectivos medios donde viven y se comunican, fueron comprometiendo a toda la red, que se fue involucrando en el trabajo, conociendo las personas que conducían este trabajo de extensión e investigación, relacionándose con ellos, discutiendo sus adelantos, manejando una nueva información de si mismos que nunca habíamos manejado. Un nuevo espejo que compromete a mejorar, continuar, trabajar…\\

Este trabajo deja memoria colectiva, permite mostrar el trabajo de comunicación en la radio comunitaria uruguaya a quienes están comprometidos en las radios, y a quienes vendrán y que podrán saber cómo somos, que buscamos, cuales son los proyectos político-comunicacionales y cuales son los caminos y estrategias que utilizamos en las radios comunitarias en la primera década del siglo XXI.\\

La radio comunitaria asume el compromiso y desafío de construir un proyecto democrático y participativo, en un marco histórico internacional donde prevalece la propiedad privada, el individualismo y el egoísmo. Con otro objetivo entonces, antagónico al “ruido” intenta escuchar, respetar, permitir la palabra de la pequeña voz, de la más oculta, de la voz amordazada, de la sustituida o avasallada por el ruido.\\

La radio comunitaria intenta comprender, aprender y construir desde lo colectivo, convencidos que el verdadero héroe no es una persona, sino que se construye entre muchos. Y este principio hace diferentes a las radios comunitarias, las aleja del ruido y la acerca a la libertad, la justicia y los derechos de todos los hombres, mujeres y niños. Las vuelve “medios” de comunicación, orejas y voces de la gente para conversar con la gente.\\

Un trabajo, una investigación, un libro realmente necesario para quienes estamos transitando este camino, para quienes lo harán más adelante, para quienes estudian estos temas, para quienes lo construyen día a día desde cada radio, cada colectivo, cada barrio o ciudad, cada rincón del Uruguay.\\

Un libro sin final, que como los mismos autores nos informan no está planteado como al principio tal vez pensaron, no es el “manual de radio”, porque han logrado que cada realidad intervenga es este trabajo concretando el concepto de extensión universitaria y logrando así un material realmente fermental que propone no uno, sino varios manuales diferentes que tienen algunos principios y objetivos comunes realmente importantes por sus características de trabajo colectivo, tolerante, generoso y comprometido con el bienestar de la sociedad, de su gente.\vspace*{1.5cm}\\


\noindent Carlos Casares, por la Mesa Nacional de AMARC\\
\noindent Montevideo, marzo de 2011