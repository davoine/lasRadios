\chapter{Fuentes de información}

En materia de información utilizada para la salida al aire, y para la preparación de los programas, se hace un uso variado tanto en el tipo de información como en la obtención de la misma.\\

Los temas a tratar surgen en general de los propios compañeros o de sugerencias de oyentes. En algunas emisoras los temas abordados al aire, se imprimen en un boletín de la radio.\\

Se especifica a continuación diversas categorías, que resumen la variedad encontrada.\\

\section{Material gráfico}

\subsection{Diarios}

De las radios sistematizadas 13 de 15 mencionan explícitamente que utilizan los medios de prensa escritos a nivel nacional como fuente de información. Disponen de medios diversos (El País, Últimas Noticias, El Espectador, La República, Brecha, La Diaria). La Diaria es mencionada por 5 de las 15, Brecha 3 de las 15, y La República por 2 de 15, mientras que los demás medios son referenciados por única vez y por diferentes radios. En algunos colectivos se reciben los periódicos por gentileza de un kiosco cercano, que otorga el material en calidad de préstamo por unas horas.\\

Asimismo, se utilizan en algunas radios medios de prensa escritos de origen internacional.\\

Son frecuentes los diarios locales como fuente de información, 5 de 15 radios mencionan utilizarlos. Se menciona boletín CCZ 17, boletín IMM, Cosmópolis, El Tejano, Jubicerro, Visión ciudadana, San José hoy, Primera hora.\\

Sin embargo, una de las radios explicita no hacer uso del medio local, por no compartir su forma de entender el periodismo gráfico, prefiriendo entonces otras fuentes acordes a su proyecto político-comunicacional.\\

\subsection{Revistas}

Algunas radios mencionan utilizar revistas varias como fuente de información, especificando las siguientes: Revista FENAPES, SEPREDI, InfoMides, Revista Embajada de Venezuela, América XXI, Caras y Caretas, 3 puntos. En otros casos, solamente se refiere al uso de revistas en general.\\

\subsection{Libros}

Se indica el uso de libros, tanto aquellos provenientes de los propios integrantes del colectivo, como de bibliotecas zonales. Se señala también el uso de enciclopedias como material de referencia.\\

Como particularidad, Horizonte de Artigas señala que acude a lanzamientos de libros, y Espika organiza presentaciones de libros.\\

\subsection{Otros}

Por otra parte se indica el uso de materiales de AMARC, la publicación ``Cara y Señal'', así como otras publicaciones de la red.\\

Se hace referencia también al uso de los almanaques del Banco de Seguros del Estado.\\

\section{Web\label{web}}

Todas las radios actualmente tienen acceso a internet (algunas lo lograron en el marco del proyecto TICs de AMARC). De todos modos, el empleo de este recurso no es utilizado de la misma forma en todos los colectivos.\\

Se utilizan programas de mensajería instantánea (chat) o redes sociales, para comunicarse entre los integrantes del colectivo y/o para comunicarse con la comunidad.\\

Algunos colectivos poseen un mayor conocimiento de páginas que son de utilidad para su proyecto político-comunicacional, otros realizan búsquedas según la necesidad y no tienen páginas de referencia o de uso frecuente.\\

Entre las mencionadas, se pueden destacar aquellas de uso frecuente para descargar música u otros materiales:\\

\begin{itemize}
 \item \href{http://www.jamendo.com}{www.jamendo.com}
 \item \href{http://www.requecheando.com}{www.requecheando.com}
 \item \href{http://www.taringa.com}{www.taringa.com}
 \item \href{http://www.4shared.com}{www.4shared.com}
 \item \href{http://www.perrerac.org}{www.perrerac.org} (descarga de música con licencia Creative Commons)
\end{itemize}

\indent Asimismo se mencionan páginas de información general o de noticias, relacionadas con temas de interés para las radios como derechos humanos, medio ambiente, violencia doméstica, drogas, pobreza, género, sexualidad, realidades barriales, información latinoamericana, entre otros:\\

\textbf{Enciclopedia virtual}:
\href{http://www.wikipedia.org}{www.wikipedia.org}\\

\textbf{Nacionales}:
\begin{itemize}
 \item \href{http://www.montevideo.com.uy}{www.montevideo.com.uy}
 \item \href{http://www.180.com.uy}{www.180.com.uy}
 \item \href{http://www.presidencia.gub.uy}{www.presidencia.gub.uy}
 \item \href{http://www.sanjosehoy.wordpress.com}{www.sanjosehoy.wordpress.com}
 \item \href{http://www.sanjosenoticias.com.uy}{www.sanjosenoticias.com.uy}
 \item \href{http://www.paysandu.org.uy}{www.paysandu.org.uy}
\end{itemize}

\textbf{De información general}:

\begin{itemize}
 \item \href{http://www.rebelion.org}{www.rebelion.org}
 \item \href{http://www.ipsnoticias.net}{www.ipsnoticias.net}
 \item \href{http://www.jakueke.com}{www.jakueke.com}
 \item \href{http://www.pressenza.org}{www.pressenza.org}
 \item \href{http://www.indymedia.org}{www.indymedia.org}
 \item \href{http://www.amnistia.org.uy}{www.amnistia.org.uy}
 \item \href{http://www.telesurtv.net}{www.telesurtv.net}
 \item \href{http://www.idealist.org}{www.idealist.org}
\end{itemize}

Se hace mención al uso de páginas gubernamentales, sitios de organizaciones y movimientos sociales latinoamericanos, pero no aparecen especificadas.\\

Se menciona en un caso el uso de web de centros científicos y de universidades, pero no se puntualiza su forma de acceso.\\

Por último, se mencionaron aquellos sitios relacionados a AMARC y a otras Radios Comunitarias y/o online:\\

\begin{itemize}
 \item \href{http://www.amarcuruguay.org}{amarcuruguay.org}/\href{http://podcast.amarcuruguay.org}{podcast.amarcuruguay.org}
 \item \href{http://www.radialistas.net}{www.radialistas.net}
 \item \href{http://www.agenciaradioweb.com}{www.agenciaradioweb.com}
 \item \href{http://www.agenciapulsar.org}{www.agenciapulsar.org}
 \item \href{http://www.comcosur.com.uy}{www.comcosur.com.uy}
 \item \href{http://www.farco.com.ar}{www.farco.com.ar}
 \item \href{http://www.radiomundoreal.fm}{www.radiomundoreal.fm}
 \item \href{http://www.elmundo.es}{www.elmundo.es}
\end{itemize}

\section{Interna}

Se indica en este ítem aquellas fuentes de información identificadas por los colectivos como información proveniente de sus propios integrantes. Esto reúne aquel material que surge de los talleres del colectivo, de las propias experiencias de vida, y de internación, en el caso especifico de Vilardevoz. Asimismo, para esta radio en particular se referencia a la información que proviene de otros actores de la  institución Hospital Vilardebó.\\

Para otros colectivos se toma también como fuente de información los estudios particulares de sus integrantes, sus propios recursos y conocimientos.\\

\section{Desde la comunidad}

Claramente el relacionamiento con la comunidad es una fuente de información para todos los colectivos. Esto va desde el “boca a boca”, el intercambio diario con los vecinos, las propias notas que los oyentes hacen llegar al local radial o vía correo electrónico.\\

Se menciona como fuente de información básica a las “prácticas de convivencia” (pueblo en sus vínculos, los pedidos de ayuda en relación a situaciones de salud, situaciones financieras, actividades de la comunidad).\\

En El Capiz se utiliza un buzón ubicado en la puerta de la radio, de donde se extrae información.\\

Por otra parte, se hace referencia tanto a la información proveniente de integrantes de la radio que no residen actualmente en la localidad (es el caso de la Heladera y particularmente en Radio Prado, donde cuentan con una compañera en Estados Unidos, que brinda información de los uruguayos inmigrantes y organizaciones sociales por la cual ella está trabajando, siendo entonces corresponsal de la radio en el exterior).\\

Se mantiene contacto con Comisiones Barriales, Comisiones Fomento, SOCAT (Servicio de Orientación, Consulta y Articulación Territorial) y diversas organizaciones sociales. También existe una comunicación fluida con las Intendencias, Juntas Locales, Centros Comunales Zonales, Alcaldías, así como con las Instituciones Educativas de la zona. De estos contactos permanentes proviene mucha de la información que sale al aire.\\

Se señala también el intercambio en algunas radios con Federación de Cooperativas, Alcohólicos Anónimos, ULOSEV (Unidad Local de Seguridad Vial), clubes de baby fútbol, sindicatos.\\

Es importante destacar que en varias emisoras las organizaciones sociales tienen su propio espacio en la radio.\\

\section{Comunicados oficiales}

Las radios reciben con asiduidad comunicados de diferente tipo para ser transmitidos a la audiencia. Éstos provienen tanto de organizaciones sociales, instituciones locales, hasta comunicados oficiales de organismos estatales. Los mencionados son la Junta Local, Ministerio de Educación y Cultura, Administración de Servicios de Salud del Estado, Administración Nacional de Telecomunicaciones (ANTEL) y URSEC.\\

Algunas radios han logrado mayor presencia pública en la comunidad, lo que repercute al momento de ser tomadas como medio de comunicación a ser tenido en cuenta.\\

Los comunicados se reciben por vía electrónica o por correspondencia.\\

\section{Otros medios de comunicación}

Se realiza intercambio con otros medios, tanto comunitarios como comerciales y se utiliza como fuente de información tanto la televisión como la radio. En algunos casos, existe un muy buen relacionamiento con las radios comerciales, señalado principalmente en las radios del interior.\\

Particularmente se mencionó a Radio Naciones Unidas\footnote{\href{http://www.unmultimedia.org/radio/spanish}{www.unmultimedia.org/radio/spanish}}, Radio Netherlands\footnote{\href{http://www.rnw.nl/espanol}{www.rnw.nl/espanol}} y Radio France Internacional\footnote{\href{http://www.espanol.rfi.fr}{www.espanol.rfi.fr}}, como medios de comunicación extranjeros.

\section{Entrevistas}

Se realizan notas a vecinos y entrevistas diversas. En algunos casos en vivo, y en otros casos son grabadas y luego salen al aire.

\section{Música}

La música emitida es en general variada. En algunos casos buscando la diversidad, en otros intentando diferenciarse de los medios comerciales, no reproducen música comercial o cumbia, priorizando la música nacional. Se indican géneros tales como: jazz, country, clásica y tango.\\

Uno de los colectivos especifica que se accede a música a través de la compra de discos en la feria vecinal y la colaboración de los oyentes que brindan material musical. En todas las emisoras se hace referencia al material que cada radialista trae para la radio. Asimismo se realizan descargas de internet, tal como se mencionó en el apartado \ref{web}.

\section{Telecomunicaciones}

Es una fuente de información presente en todos los casos, los mensajes de textos que envía la audiencia, así como las llamadas telefónicas.\\

Se implementa generalmente la salida al aire de las llamadas telefónicas.\\

Un colectivo utiliza de forma semanal skype con una corresponsal uruguaya en el extranjero.\\