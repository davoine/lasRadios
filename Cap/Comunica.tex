\chapter{Comunicación\label{Comunica}}

En este documento se presentan las diferentes estrategias de comunicación  mencionadas por los diferentes colectivos. Si bien inicialmente en la consigna se pretendió que se incluyera una evaluación sobre la funcionalidad, o los contenidos de esa información, este aspecto no fue profundizado en el material recogido. Sin embargo, éste fue tomado como insumo por la Asamblea de AMARC, donde se planteó la necesidad de conseguir dar esta discusión colectivamente, y repensar la funcionalidad de las diferentes estrategias usadas, el por qué de cada dispositivo, y los mensajes que se transmiten.\\

En relación a las estrategias de comunicación con la comunidad empleadas por los diferentes colectivos no se realiza en este documento una revisión profunda de las diferentes concepciones de relacionamiento con la comunidad o de participación comunitaria. Se presentan los diferentes mecanismos de difusión del proyecto político-comunicacional de la radio y de actividades que ésta realiza.\\

\section{Comunicación Interna}

Del material recogido se desprende que se prioriza la palabra, escuchar lo que el compañero tiene para decir, tanto en los talleres y asambleas, como en el día a día, en los aspectos organizativos. El “boca a boca”, para la transmisión de mensajes es de uso general en todas las radios.\\

Es también de uso frecuente el formato papel, tanto carteles en espacios del local radial así como carteleras. Se utilizan también otros dispositivos similares (pizarras, pizarrones) en espacios visibles del local radial, donde se indica información actualizada y relacionada tanto con actividades como con aspectos de convivencia.\\

Algunos ejemplos:\\

\textit{¿Hemos analizado la forma en que tejemos nuestra programación y nuestros programas? ¿Hemos aprovechado los recursos y potencialidades sonoras que existen en nuestra localidad para enriquecer nuestro trabajo?}

(Cartel en estudio de la Radio Horizonte, Paysandú)\\

\textit{Porque es necesario pensar en hacer otra historia, te convidamos a compartir con este colectivo. ANIMATE !!! seguro tenes algo importante pa' decir.}

(Cartelera de la Radio Utopía, Colonia Nicolich)\\

\textit{Si (el baño) hablara ¿qué diría de ti? La limpieza del baño habla de quien lo usa. El compañerismo también se mide en ello.}

(Cartel en baño del local de la Radio Horizonte, Paysandú)\\

Asimismo, a modo de registro, en las paredes de las radios hay afiches y cartelería diversa de actividades organizadas por la radio u otras organizaciones, así como también de grupos musicales y fotos de personalidades públicas referentes para el colectivo. En algunos casos hay diplomas encuadrados, o material de la red o de AMARC internacional.\\

Para actividades de la radio, en radio Gral. Artigas de Toledo, se utiliza un dispositivo de comunicación, que facilita la organización de talleres u otras actividades a realizarse colectivamente. Se realizan “Comunicados con notificación”, donde se especifica la actividad, lugar y horario, y una grilla. Allí cada integrante firma su notificación de saber la actividad, y especifica si puede participar, y en qué horario.\\

La radio misma se vuelve un medio de comunicación interna, ya que los integrantes del colectivo escuchan la radio en sus casas o trabajos. En los distintos programas se brinda información de cuestiones internas.\\

Asimismo los integrantes de las radios que permanecen más horas en las mismas (operadores, referentes de programas) se convierten en referentes en la comunicación interna.\\

Por otra parte, es de uso frecuente la telefonía como medio de comunicación entre los integrantes del colectivo. Se utilizan los teléfonos de la propia radio, los teléfonos fijos de los integrantes, así como sus celulares, tanto vía mensaje de texto o llamadas. En algunos colectivos se utiliza el correo electrónico como medio de comunicación interna, y en algún caso cuentan con una lista de distribución para los mismos.\\

Puesto que, en general, las radios están muy vinculadas al espacio local, la cercanía colabora en optimizar la comunicación: se va a la casa de los compañeros, se utiliza una suerte de comunicación “puerta a puerta”. En el caso de las radios cuyos locales se encuentran en la casa de alguno de sus integrantes, ésta sirve como espacio de encuentro (La Heladera, Radio Prado, Horizonte Max).\\

Las reuniones sistemáticas son una forma de garantizar ciertos niveles de circulación de la información relacionada a la radio y las decisiones referidas a la misma. En ocasiones, estas reuniones son semanales o quincenales, mientras que en otros la periodicidad es relativa, pudiendo ser como las mencionadas o a demanda (motivadas por asuntos de interés específico del colectivo, ingreso de nuevos integrantes).\\

Para mantener la información sistematizada se plantean ejemplos, citados como dispositivos de mucha utilidad:

\begin{itemize}
 \item \textbf{Libro de actas}. Utilizado en La Heladera, como herramienta de registro interna y con validez formal. Se deja constancia de las Asambleas, reuniones de la comisión directiva, etc.
 \item \textbf{Cuaderno de actas}. En Timbó, se deja constancia de las decisiones del colectivo en un cuaderno, pero con formalidades que le dan validez legal. Por ejemplo, se firma cruzado, cuando se anexa alguna hoja.
 \item \textbf{Actas y crónicas de las actividades realizadas}. Vilardevoz registra sus actividades en estas dos modalidades, y las sube a su web.
 \item \textbf{Cuaderno de comunicaciones}. Utilizado por Espika, donde se registra el estado del local radial al llegar, en el caso de que haya algo importante para comunicar a los compañeros que vendrán.
\end{itemize}

\section{Comunicación con la Comunidad}
En la salida al aire, se establece un contacto y comunicación permanente.\\

Se utilizan también:\\

\begin{itemize}
  \item Piques del proyecto comunicacional, o simplemente con el nombre de la emisora.
  \item Spot de programas y eventos.
  \item Spot solicitados por la propia comunidad (El Capiz). Pueden ser grabados por los propios actores implicados, o puede ser grabado por algún integrante de la radio sobre el texto que se les brinda.
\end{itemize}

Se realiza intercambio en el día a día con el vecino, a través de los diferentes programas, de correos electrónicos, del boca a boca, teléfono, mensaje de texto, encuentros casuales con la audiencia.\\

Se citan también las visitas y la participación de la audiencia en persona en los programas. Cabe destacar el singular caso de Vilardevoz, que realiza una Fonoplatea, sábado a sábado, donde se invita a participar de la salida al aire desde el Hospital Vilardebó.\\

Se mencionan asimismo salidas al barrio, para conseguir sponsor, para hablar con la audiencia, mostrar la radio.\\

Se realizan frecuentemente entrevistas a diferentes actores sociales, tanto en vivo como grabadas.\\

Por otra parte, se utilizan medios de publicidad audiovisual, realizados por los propios integrantes o a través de la  contratación de algún servicio. Se utilizan: volantes, afiches, pancartas, pegotines, pintadas en muros de la ciudad,\\

Pizarrón en la puerta del local radial, cartelería en la ruta y en el pueblo (El Capiz), publicidad rodante (Universo, La Cotorra).\\

Se utilizan medios gráficos de comunicación:\\

\begin{itemize}
  \item Boletín Sinapsis (Espika).
  \item Boletín Vilardevoz, (Vilardevoz)
  \item La Cotorra emitía un Periódico “La Jaula”, que no está ya en circulación
\end{itemize}


Otra vía de comunicación con la comunidad es a través de Internet. Varias emisoras disponen de su propia página web, donde comunican sus principios, programación, actividades y donde en algunos casos trambién se puede escuchar la radio online. Se detallan a continuación las páginas correspondientes:

\begin{itemize}
  \item El Puente: \href{http://www.lateja.org.uy/elpuente/index.html}{www.lateja.org.uy/elpuente/index.html}
  \item Prado: \href{http://www.elpradofm.net}{www.elpradofm.net}
  \item General Artigas: \href{http://radioartigas.radioteca.net}{radioartigas.radioteca.net}
  \item Horizonte: \href{http://www.horizonte989.org}{www.horizonte989.org}
  \item Horizonte Max: \href{http://horizontemax913.blogspot.com}{horizontemax913.blogspot.com}
  \item La Heladera: \href{http://laheladera.awardspace.biz/index1.html}{laheladera.awardspace.biz/index1.html}
  \item Timbó: \href{http://www.fmtimbo.ensanjose.com}{www.fmtimbo.ensanjose.com}
  \item Universo: \href{http://www.universofm.es.tl}{www.universofm.es.tl}
  \item Vilardevoz: \href{http://radiovilardevoz.wordpress.com}{radiovilardevoz.wordpress.com}
\end{itemize}

Varios colectivos utilizan sistemas de comunicación instantánea a través de internet para comunicarse con la audiencia. En algunos casos se utilizan también redes sociales, en otros colectivos el uso, el objetivo y los resultados posibles del uso de estos medios es aún objeto de discusión.\\

También se realizan diferentes tipos de eventos sociales (actividades culturales, espectáculos musicales, transmisión de Carnaval), que son considerados espacios de comunicación con la comunidad. En algunos casos a través del apoyo a actividades que ésta realiza, en otros casos desde las actividades de la propia radio.\\

En el caso de La Cotorra se participa en la elaboración y difusión de las actividades del Centro Cultural Florencio Sánchez. Esta emisora es además parte de los destinos a visitar el día del Patrimonio en el marco del Cerro Cultural.\\

Algunas emisoras festejan sus cumpleaños en algún espacio público, realizando una invitación abierta a los vecinos.\\

También se realiza apoyo solidario de actividades, o se realizan campañas de solidaridad con algún vecino. Radio Prado brinda wifi de acceso libre para las XO; Utopía lleva adelante una bolsa de trabajo.\\

Por otra parte cabe destacar que algunos colectivos han realizado proyectos de desarrollo específico con la comunidad.\\
\begin{itemize}
  \item Talleres con escuelas y liceos (La Cotorra).
  \item Primer Escuela de Rock (La Cotorra).
  \item Proyecto “Galpones”: espacio socio-cultural, cogestionado con otros organizaciones sociales que está en proceso (Espika).
  \item Aire Mojado (Fondos Concursables MEC - Espika).
  \item “El Cerro habla”: documental (Fondos PNUD-La Cotorra).
  \item “Entre jóvenes” (INJU-Insomnio)
\end{itemize}
Algunos de ellos contaron con financiamiento (ver capítulo \ref{financieros}).

En relación al contacto con otras organizaciones, se destaca la particularidad de Vilardevoz de participar en congresos, o de realizar desembarcos (presentación de la radio en otros espacios, como Facultad de Psicología).\\

Asimismo cabe destacar la participación directa en el SOCAT, como representante barrial, de La Cotorra. En este caso se establece también un acuerdo de trabajo con un pasante de comunicación del Centro Comunal Zonal 17.\\

En Impactos y Utopía hay organizaciones sindicales que tienen sus propios programas en la radio.\\

Se menciona asimismo la realización de Asambleas barriales, y otras modalidades novedosas como la participación en movilizaciones, o cortes de ruta promovidas por tema de preocupación de la comunidad (Utopía), cartelera con estado de cuenta de la radio (Universo), buzón en la puerta del local radial (El Capiz). Radio Espika ha organizado Foros Sociales en la ciudad de Santa Lucía.\\

\section{Comunicación con la Red}

En lo que respecta a la comunicación con la red se plantea la existencia de dificultades para mantener una comunicación fluida.  Se destaca que el acceso a internet de todas las radios puede generar más posibilidades de mantener un contacto más fluido. Se plantea asimismo la existencia de relacionamiento con radios comunitarias no asociadas a AMARC, en particular por cercanía geográfica.\\

Se citan la comunicación telefónica como la vía principal de comunicación. También el correo electrónico es de uso frecuente. El podcast es usado para compartir producciones con otros colectivos.\\

Se menciona asimismo las asambleas como espacio privilegiado de comunicación de la red, destacando además los espacios colectivos como los diferentes talleres de formación. Los campamentos anuales son mencionados como un espacio de comunicación importante.\\

Se hace referencia a las acciones sociales conjuntas, los dúplex entre varias emisoras, y las colaboraciones técnicas.\\

Se realizan visitas a los locales radiales de emisoras asociadas, así como intercambios entre diferentes compañeros de la mesa o las áreas de trabajo.