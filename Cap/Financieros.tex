\chapter{Recursos financieros\label{financieros}}

\section{Socios}

Se considera socio colaborador aquella persona que contribuye al sustento del colectivo mediante una aportación monetaria mensual. Es la fuente de financiamiento más genuina, por hacer parte a la comunidad en el sustento económico del proyecto radial. Resulta oportuno destacar que ser colaborador no implica ser miembro activo, aunque si puede ejercer el derecho por la propia definición de comunitaria de las radios.\\

Menos de la mitad de las radios visitadas instrumenta este sistema. Lo hacen mediante el cobro de una cuota que oscila entre los $\$$10 y $\$$50 de forma mensual o bimensual. Quien se encarga de la cobranza por lo general es un compañero del colectivo y retira para sí un porcentaje del dinero cobrado a modo de comisión.\\

En el caso de Radio Horizonte Max de Artigas cada comunicador es socio colaborador. En Radio Timbó sus integrantes colaboran también con aportes mensuales.\\

En La Cotorra se llegó a un acuerdo con una compañera del colectivo con experiencia en ventas y marketing para que se encargara de la captación y fidelización de socios.\\

\section{Publicidad}

La publicidad es una forma destinada a difundir o informar al público sobre un bien o servicio a través de los medios de comunicación con el objetivo de motivarlos hacia una acción de consumo.\\

De las radios relevadas, casi la totalidad ofrece publicidad. De lo recaudado, el principal destino es el sustento económico del colectivo, básicamente para el pago de gastos generales.\\

Por lo general, se cobran cuotas de bajo monto y son a comercios locales o “comercios amigos” a los cuales se le efectúan canjes (tanto en bienes como en servicios) o el propio cobro contado.\\

Los avisos publicitarios salen distintamente al aire dependiendo del importe que abona y de las pautas que se establecieron en cada colectivo. La modalidad instrumentada varía en cada uno de ellos, va desde quienes mencionan 6 avisos por programa hasta quienes sólo sacan 4 en el correr del día.\\

En el caso de Radio Espika, sólo se ofrece publicidad cuando se realizan eventos. Lo recaudado se destina para financiar los posibles costos de cada evento, como por ejemplo el sonido. Actualmente es una de las radios que se encuentra sin vender publicidad y viene atravesando un largo proceso de discusión al respecto. Se viene evaluando los contenidos de las publicidades y las características de cada auspiciante, ya que entienden que la publicidad que se transmite es parte del contenido de la radio.\\

La venta de publicidad es el recurso financiero que mayor discusión genera dentro del colectivo. Los proyectos de las radios comunitarias no tienen fines de lucro, lo cual no significa ir a pérdidas. Les resulta necesario lograr la sustentabilidad económica en el tiempo. Es así que surge la discusión sobre cómo debería instrumentarse este recurso financiero. Cómo vender publicidad, con qué contenido, bajo qué condiciones y quiénes serían los auspiciantes.\\

Los colectivos que hoy no venden publicidad están trabajando en ello. Unos estudiando la zona y los posibles comercios a los que se les podría vender y si bien otros ya tienen previsto el procedimiento de venta, están esperando a mejorar su programación.\\

\section{Recursos estatales}

Aunque en general se trata de proyectos financiados por entidades estatales, también existen algunos casos de trabajo con recursos de otros tipos de organismos, como el Programa Naciones Unidas para el Desarrollo (PNUD). Menos de la mitad de los colectivos han acudido a financiar algún proyecto con fondos del Estado. Es el caso de El Capiz de Valizas, Espika FM de Santa Lucía, Insomnio de Las Piedras, La Cotorra y Vilardevoz de Montevideo y Radio Horizonte de Paysandú.\\

El Capiz postuló un proyecto a los fondos vecinales de la Intendencia Municipal de Rocha. Son recursos destinados a promover el desarrollo de la región. El proyecto presentado por la radio no resultó aprobado.\\Link: \href{http://www.rocha.gub.uy}{www.rocha.gub.uy}.\\

Los compañeros de Espika presentaron en el año 2009 un proyecto a los Fondos Concursables del MEC. El mismo consistía en el trabajo conjunto de estudiantes de tercer año de liceo de las localidades de Santa Lucía, Los Cerrillos y 25 de Agosto. Los docentes participantes del proyecto donaron parte de sus honorarios a efectos de continuar con la construcción de un espacio socio-cultural.\\Link: \href{http://www.fondoconcursable.mec.gub.uy}{www.fondoconcursable.mec.gub.uy}.\\

Radio Vilardevoz trabaja actualmente con estudiantes de Facultad de Psicología, quienes realizan sus pasantías curriculares de cuarto y quinto ciclo participando en los diferentes espacios de trabajo del colectivo. En el mismo se cuenta con el apoyo de una docente de la facultad mencionada con la función de enseñanza. Asimismo, se vienen realizando algunas actividades específicas financiadas por CSEAM (Comisión Sectorial de Extensión y Actividades en el Medio), como por ejemplo el financiamiento de un desembarco en Psicología en noviembre de 2010.\\Link: \href{http://www.extension.edu.uy}{www.extension.edu.uy}.\\

La Cotorra tuvo una experiencia con el PNUD. La misma propuso como objetivo general aportar a la recuperación de la memoria reciente tomando como referente a los protagonistas del barrio (Cerro de Montevideo) intentando reflejar en ellos a la comunidad toda. Se desarrolló desde marzo a octubre 2009.\\Link: \href{http://www.undp.org.uy/pnuduruguay}{www.undp.org.uy/pnuduruguay}.\\

Insomnio postuló un proyecto al INJU (“Entre Jóvenes”) en el programa Amplifica tu Voz, el cual fue aprobado en el año 2010. El mismo tiene como objetivo general financiar el arreglo de la sede social (pintar y colocar un cartel en el frente) y conseguir una consola para la radio. \\Link: \href{http://www.inju.gub.uy}{www.inju.gub.uy}.\\

En el caso de Radio Horizonte de Paysandú, se postuló un proyecto al Presupuesto Participativo de esa localidad, el cual solicitaba híbridos, computadoras y consola. No se contó con los votos suficientes para que fuera aprobado. Entre otras estrategias, realizó una campaña vía mensajes de texto.\\Link: \href{http://www.paysandu.gub.uy}{www.paysandu.gub.uy}.\\

Existe el caso de un colectivo que no está de acuerdo con el apoyo estatal a las radios comunitarias.\\

\section{Otros}
El colectivo Espika de Santa Lucía es uno de los que más ha desarrollado la búsqueda de fondos alternativos para financiamiento.
Actualmente, cuentan con cuatro fuentes de ingresos provenientes de:
\begin{itemize}
 \item Cantina y entradas a eventos culturales
 \item Coproducciones como presentación de libros, discos y obras de teatro
 \item Distribución de La Diaria
 \item Distribución de música independiente (discografía)
\end{itemize}

\indent Casi la totalidad de los colectivos recibe una colaboración proveniente del bolsillo de cada miembro. Varias se financian también con rifas, bailes, eventos artísticos, varietés, venta de comida, canjes y bonos colaboración. Reciben asimismo donaciones puntuales (sillas, estufas o dinero para acondicionamiento del local). Hay un colectivo que se encarga de reciclar material de los lugares de trabajo de algunos de los miembros, como ser computadoras, impresoras y muebles.