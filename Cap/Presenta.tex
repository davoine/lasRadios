\phantomsection
\addcontentsline{toc}{chapter}{\protect\numberline{}Presentación}
\chapter*{Presentación}

Esta publicación es producto de un trabajo colectivo, de un equipo de estudiantes de la Universidad de la República\footnote{Portal de la Universidad de la República: \href{http://www.universidad.edu.uy}{www.universidad.edu.uy}} y muchos actores de las emisoras comunitarias asociadas a la Asociación Mundial de Radios Comunitarias de Uruguay\footnote{Página web de AMARC Uruguay: \href{http://amarcuruguay.org}{amarcuruguay.org}}.\\

Resume el esfuerzo de todo un año de trabajo, y de un intento de hacer visibles, de compartir las experiencias de 15 radios comunitarias localizadas en todo el país, de Artigas a Montevideo, de Paysandú hasta Rocha.\\

El material que publicamos es fruto de un trabajo de sistematización de experiencias diversas, y está basado en el respeto de la heterogeneidad de cada colectivo, y de sus procesos.\\

Intenta dar cuenta de los aprendizajes, las fortalezas, la potencia de los ámbitos colectivos, autogestionados en espacios de comunicación. Es una pretensión de organizar y articular saberes en esta materia, saberes acumulados por años de trabajo de hombres y mujeres que apuestan a medios de comunicación más democráticos.\\

Es un intento de dejar memoria y transmitir, de mostrar lo que se hace a los otros compañeros de la red, a los nuevos integrantes que vendrán, a los propios compañeros del colectivo, a la comunidad en general y también a la Universidad.\\

Se parte de un enfoque solidario, y que pretende que estos saberes no queden reducidos sólo a sus protagonistas, sino que sean socializados y que puedan contribuir a otros proyectos.\\

Se pretende también el reconocimiento de estos espacios colectivos e innovadores, así como comprometidos políticamente con la realidad.\\

Desde una concepción de extensión, que sigue la rica tradición del modelo latinoamericano de Universidad, que asume un compromiso profundo con la sociedad de la que es parte y a la que se debe, nuestra apuesta fue a fortalecer la red, y nos sentimos orgullosos de haber colaborado con este pequeño aporte.\\

Queremos agradecer a todos los compañeros de las radios comunitarias que participaron activamente en el proyecto, así como el apoyo brindado por la Mesa Nacional de AMARC, en particular los compañeros Victoria Méndez y Carlos Dárdano. También queremos agradecer a la Universidad de la República, y en particular al Servicio Central de Extensión y Actividades en el Medio por haber posibilitado este trabajo, a través de apoyo económico y espacios de formación, así como el apoyo brindado por Diego Castro como referente del proyecto.\\

Los responsables de este proyecto (María Noel Sosa, Ángela Garofali, Pablo Hansen y Federico Davoine) reciben comentarios, críticas y sugerencias en el correo: \href{mailto:las-radios-no-son-ruido@googlegroups.com}{las-radios-no-son-ruido@googlegroups.com}.